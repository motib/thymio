\chapter*{Preface}
\addcontentsline{toc}{chapter}{Preface}

\sect{VPL and Scratch}

VPL is a visual programming environment for the Thymio robot
(Figure~\ref{fig.vpl}). Scratch (Figure~\ref{fig.scratch}) is a
web-based visual programming environment that animates \emph{sprites} on
the computer screen. The two environments have similar features: a
program is created by dragging-and-dropping blocks onto the screen, and
the fundamental programming construct is the \emph{event handler}.
Knowledge of one environment can be helpful in learning a second
environment: we take a program written in the first environment and show
how it can be written in the second. This is based on the educational
theory called \emph{mediated transfer}; see the Wikipedia article on
\href{https://en.wikipedia.org/wiki/Transfer_of_learning}{Transfer of
learning} and the references by Salomon and Perkins cited there.

\begin{figure}[hb]
\centering
    \subfigure[VPL]{
		\label{fig.vpl}
		\includegraphics[width = 0.4\textwidth]{gui}
	}
	\hspace{1.5cm}
    \subfigure[Scratch]{
		\label{fig.scratch}
		\includegraphics[width = 0.4\textwidth]{scratch}
	}
    \caption{VPL and Scratch}
    \label{fig.vplscratch}
\end{figure}

Chapters~\ref{ch.events}--\ref{ch.projects} are about transfer from VPL
to Scratch. We take programs from the VPL tutorial \emph{First Steps in
Robotics with the Thymio-II Robot and the Aseba/VPL Environment}
(\url{https://aseba.wikidot.com/en:thymioprogram}) and show how to
implement them as Scratch programs controlling a sprite that is an image
of the Thymio robot.

Chapter~\ref{ch.brait} is intended for students with a good knowledge of
Scratch who are learning VPL. The projects from Chapter~12 of the VPL
tutorial on Braitenberg creatures have been implemented in Scratch.

\sect{References}

The document is not intended as a tutorial on Scratch, but
rather as a collection of projects that can be used when learning
Scratch. For an introduction to Scratch, I recommend \textit{Computer
Science Concepts in Scratch} by Michal Armoni and Moti Ben-Ari, which
can be downloaded for free at
\url{http://stwww.weizmann.ac.il/g-cs/scratch/scratch_en.html}.

The projects are arranged in increasing complexity of the Scratch
implementation, not in the order they appear in the VPL tutorial.

The Scratch projects described here can be found in my Thymio studio on
the Scratch website (\url{http://scratch.mit.edu/studios/1023692}),
except for the projects on the Braintenberg creatures which are in a
separate studtio (\url{https://scratch.mit.edu/studios/1452106}).

For other robotics-related Scratch projects, see my website or my
Robotics studio (\url{http://scratch.mit.edu/studios/520857}).

\sect{On the implementation}

Aside from the obvious difference between the concrete Thymio robot and
its image on the Scratch stage, the main difference between the robotic
projects and the Scratch projects is how the sensors are implemented in
Scratch. Full details of the implementation are given in
\cref{ch.implementation}, but it is not necessary to understand them,
since a set of abstractions has been introduced and they will be
described as they are introduced in the projects.

\begin{itemize}

\item A sprite called \p{Pointer} is used to sense where the mouse is
clicked. It broadcasts the messages \p{center}, \p{front}, \p{back},
\p{left}, \p{right} to simulate touching the buttons.

\item The new block \scrblk[-6]{get-pointer-direction-block} returns in the
variable \scrblk[-4]{direction-to-pointer} the direction from the
\p{Thymio} sprite to the point where the mouse was clicked.

\item The new block \scrblk[-6]{get-touching-block} returns in the variable
\scrblk{is-touching} an indication if the \p{left} or \p{right} ground
sensors are touching a black tape, or if \p{both} sensors or \p{neither}
sensor are touching the tape.

\end{itemize}

The following costumes are used:

\begin{itemize}

\item The \p{Thymio} sprite is colored violet to make it stand out on the
stage. You can change this color if you like. The buttons are colored and these
colors should not be changed.

\item Five costumes (\p{blank}, \p{red}, \p{green}, \p{blue},
\p{yellow}) simulate the top lights.

\item The costume \p{ground} simulates the ground sensors.

\end{itemize}
